\section{2-es rúd méretezése kihajlásra}

\subsection{Kritikusfeszültség - karcsúság diagram}

\begin{align*}
	\sigma_\text{F} &= \mpa{240} \\
	\lambda_0 &= 150 \\ \\
	\sigma_\text{kr}(\lambda) &= 308 - 1.14\lambda
\end{align*}

\begin{align*}
	&\sigma_\text{kr}(\lambda_0) = 188.3 \\
	&\sigma_\text{kr}(\lambda_1) = \sigma_\text{F} \Rightarrow \lambda_1 = 59.65
\end{align*}

A kezdeti szakaszon $\lambda$-tól független konstans $\sigma_\text{F}$ a kritikusfeszültség. A Tetmajer-egyenes két szélsőértéke az Euler kezdetét jelentő $\lambda_0$ valamit $\lambda_1$ ahol az előbbi egyenessel számolt feszültség eléri a folyáshatárt.
\begin{center}
        \begin{tikzpicture}
                \begin{axis}[
                        axis lines = left,
			xlabel={$\lambda [-]$},
			ylabel={$\sigma_\text{kr}(\lambda) \mpa{}$},
			ymin=0, ymax=250,
			xmin=0, xmax=200
                ]
                        \addplot [
                                domain=0:59.65,
                                red
                        ]
                                {240};
                        \addplot [
                                domain=59.65:105,
                                red
                        ]
                                {308-1.14*x};
                        \addplot [
                                domain=105:200,
                                red
                        ]
				{19771.5/x};
			\addplot[only marks, mark=*] coordinates {(59.65, 239.99)};
			\addplot[only marks, mark=*] coordinates {(105, 188.3)};
			\node at (59.65 -20, 239.99 -20) {$(59.65; 239.99)$};
			\node at (105 + 30, 188.3) {$(105; 188.3)$};
                \end{axis}
        \end{tikzpicture}
\end{center}

\subsection{Minimális falvastagság}
A minimális falvastagság számítása függ attól épp melyik tartományba esik a rúd amit csak a falvastagságból tudnánk, ezért feltételezzük hogy a rúd az Euler-tartományba esik karcsúság szempontjából. A $c$ szorzó az egyenértékű hosszhoz abból származik hogy a rúd alul be van fogva, felül pedig nincs akadályozva a kihajlás.

\begin{align*}
	c &= 2 \\
	h_0 &= ch = \m{5} \\
	F_t = 3\left|B_z\right| &= \left(\frac{\pi}{h_0}\right)^2 I_2 E 
\end{align*}

A keresztmetszet másodrendű nyomatéka függ a geometriától.
\begin{align*}
	I_2 &= \frac{d^4 \pi}{64} - \frac{(d-2t_\text{min})^4 \pi}{64} \\
	t_\text{min} &= 2.49254 \approx \mm{2.5}
\end{align*}

\begin{align*}
	A &= \frac{\left[d^2 - (d-2t_\text{min})^2\right] \pi}{4} \\
	i_2 &= \sqrt{\frac{I_2}{A}} \\ \\
	\lambda &= \frac{h_0}{i_2} = 254.886
\end{align*}
A karcsúság az Euler-tartományba esik, tehát feltételezésünk beigazolódott.
