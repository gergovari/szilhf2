\section{Lehajlásfüggvény}

\subsection{Hajlítónyomatéki igénybevételi függvény}
\newcommand{\mcolor}{cyan}
{\footnotesize
        \begin{center}
                \setlength{\aboverulesep}{0pt}
                \setlength{\belowrulesep}{0pt}
                \setlength{\extrarowheight}{.75ex}
		\begin{tabular}{rccc}
			\toprule
			\rowcolor{lightgray}
			$x$
			&$0 < x < L$
			&$L < x < L+R$
			&$L+R < x < 2L+R$ \\
			\toprule

			\rowcolor{\mcolor}
			$M_h$
			&$-M$
			&$-M-A_z(x-L)$ 
			&$-C_z(3L-x) + p\frac{(3L-x)^2}{2}$ \\

			\bottomrule
		\end{tabular}
	\end{center}
}

\pgfmathsetmacro\M{2}
\pgfmathsetmacro\Az{1.98958}
\pgfmathsetmacro\p{1.75}
\pgfmathsetmacro\Cz{4.63542}

\begin{center}
	\begin{tikzpicture}
                \begin{axis}[
			xlabel={$x \m{}$},
                        ylabel={$M_h(x) \knm{}$},
                        axis lines=left,
                ]
                        \addplot [
                                domain=0:1.5,
                                \mcolor
                        ]
                                {-\M};
                        \addplot [
				domain=1.5:2*1.5,
                                \mcolor
                        ]
                                {-\M-\Az*(x-1.5)};
                        \addplot [
                                domain=2*1.5:3*1.5,
                                \mcolor
                        ]
                                {-\Cz*((3*1.5)-x)+ \p*(((3*1.5)-x)^2)/2};
                \end{axis}
        \end{tikzpicture}
\end{center}

\newpage

\subsection{Rugalmas szál differenciálegyenlete}
\begin{equation*}
	w_i^{''}(x) = -\frac{M_{h_i}(x)}{IE}
\end{equation*}

Mivel a hajlítónyomatéki függvényünk három részből áll, itt is követnünk kell ezt.

\begin{align*}
	w_1^{''}(x) &= 0.0122414 \\
	w_2^{''}(x) &= -0.0602503 + 0.0121776x \\
	w_3^{''}(x) &= 0.0192229 + 0.0198285x - 0.00535561x^2 \\ \\
	w_1^{'}(x) &= c_1 + 0.0122414x \\
	w_2^{'}(x) &= c_3  -0.00602503x + 0.00608881x^2 \\
	w_3^{'}(x) &= c_5 + 0.0192229x+0.00991425x^2-0.0017852x^3 \\ \\
	w_1(x) &= c_2 + c_1 x + 0.0061207x^2 \\
	w_2(x) &= c_4 + c_3 x - 0.00301252x^2 + 0.0020296x^3 \\
	w_3(x) &= c_6 + c_5 x + 0.00961146x^2 + 0.00330475x^3 - 0.000446301x^4
\end{align*}

\pgfmathsetmacro\cone{-0.049647}
\pgfmathsetmacro\ctwo{0.0606989}
\pgfmathsetmacro\cthree{-0.0359472}
\pgfmathsetmacro\cfour{0.053849}
\pgfmathsetmacro\cfive{0.0979194}
\pgfmathsetmacro\csix{0.127812}

\begin{center}
	\begin{tikzpicture}
                \begin{axis}[
			xlabel={$x \m{}$},
			ylabel={$w_i(x) \mm{}$},
                        axis lines=left,
                ]
                        \addplot [
                                domain=0:1.5,
                                \mcolor
                        ]
                                {\ctwo + \cone * x + 0.0061207*x^2};
                        \addplot [
				domain=1.5:2*1.5,
                                \mcolor
                        ]
                                {\cfour + \cthree * x - 0.00301252*x^2 + 0.00202969*x^3};
                        \addplot [
                                domain=2*1.5:3*1.5,
                                \mcolor
                        ]
			{\csix - \cfive * x + 0.00961146*x^2 + 0.00330475*x^3 - 0.000446301 * x^4};
			\addplot[only marks, mark=*] coordinates {(0, 0.0606989)};
			\addplot[only marks, mark=*] coordinates {(1.5, 0)};
			\addplot[only marks, mark=*] coordinates {(1.5*2, -0.0263059)};
			\addplot[only marks, mark=*] coordinates {(1.5*3, -0)};
                \end{axis}
        \end{tikzpicture}
\end{center}

\newpage

\subsubsection{Peremfeltételek}

Tudjuk hogy $x = L$ és $x = 3L$-ben a lehajlás értéke zérus, valamint a rúd folytonossága miatt feltételezzük hogy az egybeeső pontok érték az elhajlásnál és szögelfordulásnál megegyezik

\begin{align*}
	w_1(L) &= 0 \\
	w_2(L) &= 0 \\
	w_3(3L) &= 0 \\
	w_1^{'}(L) &= w_2^{'}(L) \\
	w_2^{'}(2L) &= w_3^{'}(2L) \\
	w_2(2L) &= w_3(2L) 
\end{align*}

\begin{align*}
	&c_1 = -0.049647 \\
	&c_2 = 0.0606989 \\
	&c_3 = -0.0359472 \\
	&c_4 = 0.053849 \\
	&c_5 = 0.0979194 \\
	&c_6 = 0.127812 \\
\end{align*}

\begin{align*}
	w_1(0) &= 0.0606989 \\
	w_1(L) &= 0 \\
	w_2(L) &= 0 \\
	w_2(2L) &= -0.0263059 \\
	w_3(2L) &= -0.0263059 \\
	w_3(3L) &= 0 \\
\end{align*}

Lehajlásfüggvény szélsőértéke ott lehet ahol a szögelfordulás nulla: a rúd szélén.

\begin{align*}
	w_{\text{max}} &= \m{0.06} = \mm{60} \\
	x_{\text{max}} &= \m{0}
\end{align*}

\newpage

\subsubsection{Szögelfordulás}

\begin{equation*}
	\phi_i(x) = w_i^{'}(x)
\end{equation*}

\begin{align*}
	\phi_1(0) &= -0.049647 \\
	\phi_2(L) &= -0.0312849 \\
	\phi_2(2L) &= 0.000777027 \\
	\phi_3(3L) &= 0.0266705
\end{align*}

\begin{center}
	\begin{tikzpicture}
                \begin{axis}[
			xlabel={$x \m{}$},
			ylabel={$\phi_i(x) \mm{}$},
                        axis lines=left,
                ]
                        \addplot [
                                domain=0:1.5,
                                red
                        ]
                                {\cone + 0.0122414*x};
                        \addplot [
				domain=1.5:2*1.5,
                                red
                        ]
                                {\cthree - 0.00602503*x + 0.00608881*x^2};
                        \addplot [
                                domain=2*1.5:3*1.5,
                                red
                        ]
			{-\cfive + 0.0192229*x + 0.00991425*x^2-0.0017852*x^3};
			\addplot[only marks, mark=*] coordinates {(0, -0.049647)};
			\addplot[only marks, mark=*] coordinates {(1.5, -0.0312849)};
			\addplot[only marks, mark=*] coordinates {(1.5*2, 0.000777027)};
			\addplot[only marks, mark=*] coordinates {(1.5*3, 0.0266705)};
                \end{axis}
        \end{tikzpicture}
\end{center}
