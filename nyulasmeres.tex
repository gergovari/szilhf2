\section{Nyúlásmérés}

\subsection{Alakváltozási tenzor}
\begin{align*}
	\epsilon_x &= \epsilon_a \\
	\epsilon_z &= \epsilon_c \\
	\epsilon_y &= -\frac{\nu}{1-\nu}(\epsilon_x + \epsilon_z)
\end{align*}

\begin{equation*}
	\pmb{\epsilon} = \begin{bmatrix}
		\epsilon_x & 0 & \frac{1}{2}\gamma_{xz} \\
		0 & \epsilon_y & 0 \\
		\frac{1}{2}\gamma_{zx} & 0 & \epsilon_z \\
	\end{bmatrix} = \begin{bmatrix}
		\num{-5.2e-4} & 0 & \num{-3.4e-4} \\
		0 & \num{9.42857e-5} & 0 \\
		\num{-3.4e-4} & 0 & \num{3e-4} \\
	\end{bmatrix}
\end{equation*}

\begin{equation*}
	\epsilon_\text{I} = \frac{\Delta V}{V} = \text{tr}(\pmb{\epsilon}) = \epsilon_x + \epsilon_y + \epsilon_z = \num{-1.257143e-4}
\end{equation*}

\subsection{Hooke-törvény}

\begin{equation*}
	\pmb{\sigma} = \begin{bmatrix}
		\sigma_x & 0 & \tau_{xz} \\
		0 & 0 & 0 \\
		\tau_{zx} & 0 & \sigma_z
	\end{bmatrix} = \frac{E}{1+\nu}\left(\pmb{\epsilon} + \frac{\nu}{1-2\nu} \epsilon_I \pmb{E}\right)
\end{equation*}

\begin{align*}
	\sigma_x &= \frac{E}{1-2\nu}\left(\epsilon_x + \frac{\nu}{1-2\nu} \epsilon_I\right) \\
	\sigma_z &= \frac{E}{1-2\nu}\left(\epsilon_z + \frac{\nu}{1-2\nu} \epsilon_I\right) \\
	\tau_{xz} &= \tau_zx = \frac{E}{1+\nu} \frac{1}{2}\gamma_xz
\end{align*}

\begin{equation*}
	\pmb{\sigma} = \begin{bmatrix}
		-99.231 & 0 & -54.9231 \\
		0 & 0 & 0 \\
		-54.9231 & 0 & 33.231
	\end{bmatrix} \left[\si{\mega\pascal}\right]
\end{equation*}

\begin{align*}
	\sigma_\text{I} &= \text{tr}(\pmb{\sigma}) = \mpa{-66} \\
	\sigma_\text{II} &= \sigma_x \sigma_y + \sigma_x \sigma_z + \sigma_y \sigma_z - \tau_{xy}^2 - \tau_{xz}^2 - \tau_{yz}^2 \\&= \SI{-6314.092275}{[\mega\pascal^2]}\\
	\sigma_\text{III} &= \text{det}(\pmb{\sigma}) = 0
\end{align*}
