\documentclass{article}

\usepackage[provide=*, magyar]{babel}

\usepackage{graphicx}
\usepackage{tikz}
\usepackage{pgfplots}

\pgfplotsset{compat=1.18}
\usetikzlibrary{calc}
\usepackage{calc}

\usetikzlibrary {arrows.meta}
\usepgfplotslibrary{fillbetween}

\usepackage{pdfpages}

\usepackage{amsmath}
\usepackage{siunitx}
\usepackage{tabularx}
\usepackage{booktabs}
\usepackage[table]{xcolor}
\usepackage{multicol}

\newcommand{\siunit}[2]{
	\SI{#1}{[#2]}
}

\newcommand{\n}[1]{
	\siunit{#1}{\newton}
}
\newcommand{\nmm}[1]{
	\siunit{#1}{\newton\mm}
}
\newcommand{\kn}[1]{
	\siunit{#1}{\kilo\newton}
}
\newcommand{\knm}[1]{
	\siunit{#1}{\kilo\newton\meter}
}
\newcommand{\mpa}[1]{
	\siunit{#1}{\mega\pascal}
}

\newcommand{\m}[1]{
	\siunit{#1}{\m}
}
\newcommand{\mm}[1]{
	\siunit{#1}{\mm}
}

\newcommand{\equal}[2]{
	\sum{#1} := 0 = #2
}

\newcommand{\circled}[1]{
	\raisebox{.5pt}{\textcircled{\raisebox{-.9pt} {#1}}}
}


\title{Szilárdságtan HF2}
\date{\today}
\author{Vári Gergő}

%\pgfmathsetmacro\s{.002}

\pgfmathsetmacro\d{58}
\pgfmathsetmacro\ds{\d * \s * 50}

\pgfmathsetmacro\L{1500}
\pgfmathsetmacro\Ls{\L * \s}

\pgfmathsetmacro\h{2500}
\pgfmathsetmacro\hs{\h * \s}

\newcommand{\forcewidth}{2}
\newcommand{\forcelength}{1.5}

\newcommand{\structurecolor}{lightgray}
\newcommand{\coordcolor}{orange}
\newcommand{\normalforcecolor}{blue}
\newcommand{\sharedforcecolor}{red}
\newcommand{\reactionforcecolor}{violet}
\newcommand{\beamforcecolor}{olive}

\newcommand{\coords}{
	\coordinate (A) at (\Ls, 0);
	\coordinate (B) at (3 * \Ls, -\hs);
	\coordinate (C) at (3 * \Ls, 0);
}

\newcommand{\coordsize}{4pt}
\newcommand{\pointsone}{
	\fill[\coordcolor] (A) circle (\coordsize) node[below right] {$A$};
	\fill[\coordcolor] (C) circle (\coordsize) node[below right] {$C$};
}
\newcommand{\pointstwo}{
	\fill[\coordcolor] (B) circle (\coordsize) node[below right] {$B$};
	\fill[\coordcolor] (C) circle (\coordsize) node[below right] {$C$};
}
\newcommand{\points}{
	\pointsone
	\pointstwo
}

\newcommand{\beamone}{
	\draw[line width=\ds, \structurecolor] (0, 0) -- (A);
	\draw[line width=\ds, \structurecolor] (A) -- (C);
	\fill[\structurecolor] (C) circle (.1);
}

\newcommand{\beamtwo}{
	\fill[\structurecolor] (C) circle (.1);
	\draw[line width=\ds, \structurecolor] (C) -- (B);
}

\newcommand{\sizewidth}{.1}
\newcommand{\sizelength}{6}
\newcommand{\sizesone}{
	\draw[line width=\sizewidth] (0, 0) -- +(0, \sizelength * 0.35);
	\draw[line width=\sizewidth] (A) -- +(0, \sizelength * 0.35);
	\draw[line width=\sizewidth] (2 * \Ls, 0) -- +(0, \sizelength * 0.35);
	\draw[line width=\sizewidth] (C) -- +(0, \sizelength * 0.35);

        \draw[line width=\sizewidth, Stealth-Stealth, ]
                (0, \sizelength * 0.35) -- +(\Ls, 0)
                node[midway, above] {$L$};
        \draw[line width=\sizewidth, Stealth-Stealth, ]
                (\Ls, \sizelength * 0.35) -- +(\Ls, 0)
                node[midway, above] {$L$};
        \draw[line width=\sizewidth, Stealth-Stealth, ]
                (2 * \Ls, \sizelength * 0.35) -- +(\Ls, 0)
                node[midway, above] {$L$};

}
\newcommand{\sizestwo}{
	\draw[line width=\sizewidth] (C) -- +(\sizelength * 0.35, 0);
	\draw[line width=\sizewidth] (B) -- +(\sizelength * 0.35, 0);
        \draw[line width=\sizewidth, Stealth-Stealth, ]
                (3*\Ls + \sizelength * 0.35, 0) -- +(0, -\hs)
                node[midway, right] {$h$};
}
\newcommand{\sizes}{
	\sizesone
	\sizestwo
}
\newcommand{\sizepreview}{
	\draw[line width=\sizewidth, Bar-Bar] (10, 3) -- +(1000 * \s, 0) node[midway, above] {$\m{1}$};
}

\newcommand{\convention}{
        \draw[-Stealth] 
                (1.5 * \Ls + \Ls, 1.25 * \Ls) -- +(1, 0)
                node [below] {$x$};
        \draw[-Stealth] 
                (1.5 * \Ls + \Ls, 1.25 * \Ls) -- +(0, 1)
                node [left] {$y$};

        \draw[-Stealth]
		(2 * \Ls + \Ls, 1.25 * \Ls) arc (0:180:-.5)
                node [midway, above] {$+$};
}

\newcommand{\wholestructure}{
	\begin{figure}[hbt!]
		\centering
		\begin{tikzpicture}
			\coords
			
			\sizepreview
			\sizes

			\beamone
			\beamtwo

			\points
		\end{tikzpicture}
		\caption{Léptékhelyes ábra}
        \end{figure}
}

\newcommand{\normalforcesone}{
        \draw[line width=\forcewidth, \beamforcecolor, -Stealth] 
                (A) -- +(0, \forcelength)
                node[midway, left] {$A_z$};
        \draw[line width=\forcewidth, \beamforcecolor, -Stealth] 
		(2*\Ls, 0) -- +(0, -\forcelength)
                node[midway, right] {$F$};
		\draw[line width=\forcewidth * .5, \beamforcecolor, -Stealth] 
		(0, -.5) arc (90:-75:-.5) node[midway, left] {$M$};
}
\newcommand{\reactionforcesone}{
        \draw[line width=\forcewidth, \reactionforcecolor, -Stealth] 
		(C) -- +(0, \forcelength)
                node[midway, right] {$C_z$};
        \draw[line width=\forcewidth, \reactionforcecolor, -Stealth] 
		(C) -- +(\forcelength, 0)
                node[near end, below] {$C_x$};
}
\newcommand{\sharedforces}{
        \fill[\sharedforcecolor, opacity=.4] (2*\Ls, 0.1) rectangle +(\Ls, 1);
        \draw[line width=.2, \sharedforcecolor, -Stealth] 
                (2*\Ls + 1, \Ls * 0.25) -- +(0, -.5)
                node[midway, left] {$p$};

}

\newcommand{\normalforcestwo}{
	\draw[line width=\forcewidth * .5, \beamforcecolor, -Stealth] 
		(3*\Ls, -\hs +.5) arc (-90:90:-.5) node[near end, left] {$M^B$};
}
\newcommand{\reactionforcestwo}{
        \draw[line width=\forcewidth, \reactionforcecolor, -Stealth] 
		(3*\Ls, \forcelength) -- (C)
                node[midway, right] {$C_z$};
        \draw[line width=\forcewidth, \reactionforcecolor, -Stealth] 
		(C) -- +(-\forcelength, 0)
                node[near end, below] {$C_x$};
        \draw[line width=\forcewidth, \reactionforcecolor, -Stealth] 
		(B) -- +(\forcelength, 0)
                node[near end, below] {$B_x$};
        \draw[line width=\forcewidth, \reactionforcecolor, -Stealth] 
		(3*\Ls, -\hs - \forcelength) -- (B)
                node[near start, left] {$B_z$};
}

\newcommand{\forcesone}{
	\sharedforces
	\normalforcesone
	\reactionforcesone
}
\newcommand{\forcestwo}{
	\normalforcestwo
	\reactionforcestwo
}

\newcommand{\sztaone}{
	\begin{figure}[H]
		\centering
		\begin{tikzpicture}
			\coords

			\convention
			
			\sizesone

			\beamone
		
			\forcesone
			\pointsone
		\end{tikzpicture}
		\caption{1-es rúd SZTÁ}
        \end{figure}
}

\newcommand{\sztatwo}{
	\begin{figure}[H]
		\centering
		\begin{tikzpicture}
			\coords
			
			\sizestwo

			\beamtwo
		
			\forcestwo
			\pointstwo
		\end{tikzpicture}
		\caption{2-es rúd SZTÁ}
        \end{figure}
}
%

\begin{document}
	\pagenumbering{gobble}
	
	\includepdf[pages={1}, pagecommand={
	\begin{picture}(0,0) 
		\put(270, 93){Vári Gergő}
		{\fontfamily{cmr}\selectfont\put(290, 23){\Large{Vári Gergő}}}
		\put(-58, -393){\Large{1.98958}}
		\put(25, -393){\Large{0}}
		\put(75, -393){\Large{60.699}}
		\put(150, -393){\Large{2.5}}
		\put(208, -393){\Large{0.943}}
		\put(276, -393){\Large{-6.8}}
		\put(333, -393){\Large{-99.231}}
		\put(-55, -450){\Large{33.231}}
		\put(7, -450){\Large{-54.923}}
		\put(75, -450){\Large{53.041}}
		\put(155.5, -450){\Large{0}}
		\put(198, -450){\Large{-119.041}}
		\put(269, -450){\Large{19.445}}
		\put(337, -450){\Large{0.048}}
		\put(-67, -510){\Large{0.3393}}
		\put(0, -510){\Large{0}}
		\put(37, -510){\Large{-0.941}}
		\put(104, -510){\Large{0}}
		\put(155.5, -510){\Large{1}}
		\put(207, -510){\Large{0}}
		\put(247, -510){\Large{0.941}}
		\put(310, -510){\Large{0}}
		\put(346, -510){\Large{0.3393}}
	\end{picture}
}]{misc/exercise.pdf}


	\maketitle
	\rule{0pt}{50pt}
	\begin{figure}[hbt!]
		\centering
		\includegraphics[scale=1.75]{./images/cauchy_stress_components.png}
		\caption{Cauchy feszültségi tenzor}
	\end{figure}

	\newpage
	\pagenumbering{arabic}
	
	\section{Reakció komponensek}

\subsection{Léptékhelyes ábra}

\subsection{SZTÁ}

\subsection{Egyensúlyi egyenletek}

	\newpage

	\section{Lehajlásfüggvény}

\subsection{Hajlítónyomatéki igénybevételi függvény}
\newcommand{\mcolor}{cyan}
{\footnotesize
        \begin{center}
                \setlength{\aboverulesep}{0pt}
                \setlength{\belowrulesep}{0pt}
                \setlength{\extrarowheight}{.75ex}
		\begin{tabular}{rccc}
			\toprule
			\rowcolor{lightgray}
			$x$
			&$0 < x < L$
			&$L < x < L+R$
			&$L+R < x < 2L+R$ \\
			\toprule

			\rowcolor{\mcolor}
			$M_h$
			&$-M$
			&$-M-A_z(x-L)$ 
			&$-M-A_z(x-L)+F(x-2L)+p(x-2L)\frac{x-2L}{2}$ \\

			\bottomrule
		\end{tabular}
	\end{center}
}


\subsection{Rugalmas szál differenciálegyenlete}
\begin{equation*}
	w_i^{''}(x) = -\frac{M_{h_i}(x)}{IE}
\end{equation*}

\subsubsection{Peremfeltételek}
\begin{align*}
	w_1(L) &= 0 \\
	w_2(L) &= 0 \\
	w_3(3L) &= 0 \\
	w_1^{'}(L) &= w_2^{'}(L) \\
	w_2^{'}(2L) &= w_3^{'}(2L) \\
	w_2(2L) &= w_3(2L) 
\end{align*}

\begin{align*}
	&c_1 = -0.049647 \\
	&c_2 = 0.0606989 \\
	&c_3 = -0.0359472 \\
	&c_4 = 0.053849 \\
	&c_5 = 0.0979194 \\
	&c_6 = 0.127812 \\
\end{align*}

\begin{align*}
	w_1(L) &= 0.0606989 \\
	w_1(L) &= 0 \\
	w_2(L) &= 0 \\
	w_2(2L) &= -0.0263059 \\
	w_3(2L) &= -0.0263059 \\
	w_3(3L) &= 0 \\
\end{align*}

\begin{align*}
	w_{\text{max}} &= \m{0.06} = \mm{60} \\
	x_{\text{max}} &= \m{0}
\end{align*}

\subsubsection{Szögelfordulás}

\begin{equation*}
	\phi_i(x) = w_i^{'}(x)
\end{equation*}

\begin{align*}
	\phi_1(0) &= -0.049647 \\
	\phi_2(L) &= -0.0312849 \\
	\phi_2(2L) &= 0.000777027 \\
	\phi_3(3L) &= 0.0266705
\end{align*}

	\newpage

	\section{2-es rúd méretezése kihajlásra}

\subsection{Kritikus feszültség - karcsúság diagram}
\begin{align*}
	\sigma_\text{F} &= \mpa{240} \\
	\lambda_0 &= 150 \\ \\
	\sigma_\text{kr}(\lambda) &= 308 - 1.14\lambda
\end{align*}

\begin{align*}
	&\sigma_\text{kr}(\lambda_0) = 188.3 \\
	&\sigma_\text{kr}(\lambda_1) = \sigma_\text{F} \Rightarrow \lambda_1 = 59.65
\end{align*}

\subsection{Minimális falvastagság}
\begin{align*}
	c &= 2 \\
	h_0 &= ch = \m{5} \\
	F_t = 3\left|B_z\right| &= \left(\frac{\pi}{h_0}\right)^2 I_2 E 
\end{align*}

\begin{align*}
	I_2 &= \frac{d^4 \pi}{64} - \frac{(d-2t_\text{min})^4 \pi}{64} \\
	t_\text{min} &= 2.49254 \approx \mm{2.5}
\end{align*}

\begin{align*}
	A &= \frac{\left[d^2 - (d-2t_\text{min})^2\right] \pi}{4} \\
	i_2 &= \sqrt{\frac{I_2}{A}} \\ \\
	\lambda &= \frac{h_0}{i_2} = 254.886
\end{align*}

	\newpage

	\section{Nyúlásmérés}

\subsection{Alakváltozási tenzor}
\subsection{Hooke-törvény}

	\newpage

	\section{Főfeszültségek}

\subsection{Mohr-féle diagram}

\subsection{Főirányok}

\subsection{Ellenőrzés}

	\newpage

	\section{Pontbeli feszültségi állapot}

\begin{align*}
	\sigma_e^\text{Mohr} &= \sigma_1 - \sigma_3 = \mpa{172.08244}  \\
	\sigma_e^\text{HMH} &= \sqrt{\frac{1}{2}\left[(\sigma_1 - \sigma_2)^2 + (\sigma_1 - \sigma_3)^2 + (\sigma_2 - \sigma_3)^2\right]} = \mpa{152.6377}
\end{align*}
	
\begin{equation*}
	\Delta \sigma_e = \sigma_e^\text{Mohr} - \sigma_e^\text{HMH} = \mpa{19.445}
\end{equation*}

	\newpage

	\section{Pontbeli alakváltozási energiasűrűség}

\begin{equation*}
	u = \frac{1}{2}\pmb{\sigma}:\pmb{\epsilon} = \SI{0.04946}{[\joule\per\centi\meter^3]} \\
\end{equation*}

\begin{align*}
	\pmb{\epsilon_I} &= \frac{1}{3}\epsilon_I\pmb{E} \\
	\pmb{\sigma} &= \frac{1}{3}\sigma_I\pmb{E} \\
	u_h = \frac{1}{2}\pmb{\sigma_I}:\pmb{\epsilon_I} &= \frac{1}{6} \sigma_I \epsilon_I = \num{1.3828573e-3}
\end{align*}

\begin{equation*}
	u_d = u - u_h = \SI{0.0481}{[\joule\per\centi\meter^3]} \\
\end{equation*}

	\newpage
\end{document}
